\documentclass{article}
\usepackage[utf8]{inputenc}
\usepackage{amsthm}
\usepackage{amsmath}
\usepackage{amssymb}
\usepackage{mathrsfs}
\usepackage[T1]{fontenc}
\usepackage{esint}

\title{Notes on Evans PDE, Chapter 5.9}
\author{Yunzhe Zheng}
\date{2025.7}

\begin{document}

\maketitle

\section{Other Spaces of Functions}

\indent\indent This section requires basic knowledge of Functional Analysis. I will include necessary definitions and theorems in the note. You can find more in the appendix, but not all of them are needed here, and I shall include proofs of some statements there.

\subsection{Space of $H^{-1}$}
\indent\indent The construction of $H^{-1}$ requires the following definitions.
\newtheorem{Def}{Definition}[section]

\subsubsection{Elements of Functional Analysis}
\begin{Def}
    A bounded linear operator $u^*: X\to\mathbb{R}$ is called a bounded linear functional on $X$. Here $X$ is a Banach Space.
\end{Def}

\begin{Def}
    We write $X^*$ to denote the collection of all bounded linear functionals on $X$, called the Dual Space of $X$.
\end{Def}

\begin{Def}
    If $u\in X$, $u^*\in X^*$, we write $\langle u^*, u\rangle$ to denote the real number $u^*(u)$.
\end{Def}

\subsubsection{$H^{-1}$}

\begin{Def}
    $H^{-1}$ is the dual space to $H_{0}^1$, and its norm is defined as $\sup\left\{\langle f,u\rangle: u\in H^{1}_{0}, \|u\|_{H_{0}^{1}(U)\leq 1}\right\}$
\end{Def}

\subsubsection{Characterization of $H^{-1}$}

We observe one way of characterizing $H^{-1}$ by $\mathscr{L}^2$ functions.

\newtheorem{Th}{Theorem}
\begin{Th}
    (i) Assume that $f\in H^{-1}(U)$. Then there exist functions $f^0, f^1, \dots, f^n$ in $\mathscr{L}^2(U)$ such that 
    $$
    \langle f,v\rangle =\int_{U}f^0v+\sum\limits_{i=1}^{n}f^iv_{x_i}dx
    $$ \\
    (ii) Furthermore, 
    $$
    \|f\|_{H^{-1}(U)}=\inf\left\{\left(\int_{U}\sum\limits_{i=1}^n\left|f^i\right|^2\right)^{1/2}: f\text{ satisfies previous relation}\right\}
    $$\\
    (iii) In particular, we have 
    $$
    (u,v)_{\mathscr{L}^2(U)}=\langle u,v\rangle
    $$ for all $u\in H^{1}_{0}(U)$, $v\in\mathscr{L}^2(U)\subset H^{-1}(U)$.
\end{Th}

\begin{proof}
    
\end{proof}

\end{document}