\documentclass{article}
\usepackage[utf8]{inputenc}
\usepackage{amsthm}
\usepackage{amsmath}
\usepackage{amssymb}
\usepackage{mathrsfs}
\usepackage[T1]{fontenc}
\usepackage{esint}

\title{Notes on Evans PDE, Chapter 5.9}
\author{Yunzhe Zheng}
\date{2025.7}

\begin{document}

\maketitle

\section{Other Spaces of Functions}

\indent\indent This section requires basic knowledge of Functional Analysis. I will include necessary definitions and theorems in the note. You can find more in the appendix, but not all of them are needed here, and I shall include proofs of some statements there.

\subsection{Space of $H^{-1}$}
\indent\indent The construction of $H^{-1}$ requires the following definitions.
\newtheorem{Def}{Definition}[section]

\subsubsection{Elements of Functional Analysis}
\begin{Def}
    A bounded linear operator $u^*: X\to\mathbb{R}$ is called a bounded linear functional on $X$. Here $X$ is a Banach Space.
\end{Def}

\begin{Def}
    We write $X^*$ to denote the collection of all bounded linear functionals on $X$, called the Dual Space of $X$.
\end{Def}

\begin{Def}
    If $u\in X$, $u^*\in X^*$, we write $\langle u^*, u\rangle$ to denote the real number $u^*(u)$.
\end{Def}

\subsubsection{$H^{-1}$}

\begin{Def}
    $H^{-1}$ is the dual space to $H_{0}^1$, and its norm is defined as $\sup\left\{\langle f,u\rangle: u\in H^{1}_{0}, \|u\|_{H_{0}^{1}(U)\leq 1}\right\}$
\end{Def}

\subsubsection{Characterization of $H^{-1}$}

We observe one way of characterizing $H^{-1}$ by $\mathscr{L}^2$ functions.

\newtheorem{Th}{Theorem}
\begin{Th}
    (i) Assume that $f\in H^{-1}(U)$. Then there exist functions $f^0, f^1, \dots, f^n$ in $\mathscr{L}^2(U)$ such that 
    $$
    \langle f,v\rangle =\int_{U}f^0v+\sum\limits_{i=1}^{n}f^iv_{x_i}dx
    $$ \\
    (ii) Furthermore, 
    $$
    \|f\|_{H^{-1}(U)}=\inf\left\{\left(\int_{U}\sum\limits_{i=1}^n\left|f^i\right|^2\right)^{1/2}: f\text{ satisfies previous relation}\right\}
    $$\\
    (iii) In particular, we have 
    $$
    (u,v)_{\mathscr{L}^2(U)}=\langle u,v\rangle
    $$ for all $u\in H^{1}_{0}(U)$, $v\in\mathscr{L}^2(U)\subset H^{-1}(U)$.
\end{Th}

\begin{proof}
    The essence of the proof is to define an inner product for $u,v\in H_0^1(U)$
    $$
        (u, v)=\int_{U}(Du\cdot Dv+uv)dx.
    $$ and apply the Riesz Representation Theorem. The rest please refer to textbook. \\
    \indent But there are several calculation details that worth more elaborations. First in the second bullet point of the proof, after setting $v=u$, the author state without much explanation that 
    $$
        \int_{U}(|Du|^2+|u|^2)dx\leq \int_{U}\sum\limits_{i=0}^n|g^i|^2dx
    $$
    To see why, we apply Cauchy-Schwarz inequality
    \begin{align*}
        \langle f,u\rangle =\int_U g^0u + \sum\limits_{i=1}^ng^iu_{x_i}dx=\int_U\textbf{g}\cdot\textbf{u}dx\leq\int_U (|\textbf{g}|^2+|\textbf{u}|^2)/2dx
    \end{align*}
    where $\textbf{g}=(g^0, g^1, \dots, g^n)$ and $\textbf{u}=(u, u_{x_1}, \dots, u_{x_n})$. Combining withe the fact that $\langle f,v\rangle=\int_{U}(|Du|^2+|u|^2)dx$, we obtain the desired inequality. \\
    \indent Secondly, in the third bullet point of the proof, Evans claimed that 
    $$
    \left|\langle f, v\rangle\right|\leq \left(\int_U\sum\limits_{i=0}^n|f^i|^2dx\right)^{1/2}
    $$
    The reason is as follows
    \begin{align*}
        \left|\langle f, v\rangle\right|&=\left|\int_U\left(f^0v+\sum\limits_{i=1}^{n}f^iv_{x_i}\right)dx\right|\leq \left|\int_Uf^0vdx\right|+\sum\limits_{i=1}^{n}\left|\int_{U}f^iv_{x_i}dx\right| \\
        &\leq \left(\int_U|f^0|^2dx\right)^{1/2}\left(\int_U|v|^2dx\right)^{1/2}+\sum\limits_{i=1}^{n}\left(\int_U|f^i|^2dx\right)^{1/2}\left(\int_U|v_{x_i}|^2dx\right)^{1/2} \\
        &= \sum\limits_{i=0}^{n}\|f^i\|_{\mathscr{L}^2}\cdot\|v_{x_i}\|_{\mathscr{L}^2}
    \end{align*}
    by Cauchy-Schwarz inequality for $\mathscr{L}^2$ functions.
    \begin{align*}
        \left|\langle f, v\rangle\right| &\leq \left(\sum\limits_{k=0}^n\int_U|f^k|^2dx\right)^{1/2}\cdot\left(\|v\|_{\mathscr{L}^2}+\sum\limits_{i=1}^n\|v_{x_i}\|_{\mathscr{L}^2}\right)^{1/2} \\
        &= \left(\int_{U}\sum\limits_{k=0}^n|f^k|^2dx\right)^{1/2}\cdot \|v\|_{H_0^1(U)} \\
        &\leq \left(\int_{U}\sum\limits_{k=0}^n|f^k|^2dx\right)^{1/2}
    \end{align*} whenever $\|v\|_{H_0^1(U)}\leq 1$ \\
    \indent Finally, in order to deduce formula got $H^{-1}$ norm, Evans says it can be done by setting $v=\frac{u}{\|u\|_{H_0^1(U)}}$. To be more specific, we calculate
    \begin{align*}
        \left\langle f,\frac{u}{\|u\|_{H_o^1}}\right\rangle=\int_U \frac{Du\cdot Du}{\|u\|_{H_0^1}}+\frac{u^2}{\|u\|_{H_0^1}}dx=\frac{\|u\|^2_{H_0^1}}{\|u\|_{H_0^1}}=\|u\|^2_{h_0^1}
    \end{align*} thus by definition, $\|f\|_{H^-{-1}}\geq \|u\|_{H^1_0(U)}$.
\end{proof}

\subsection{Spaces involving time}

\indent\indent In this subsection we study Sobolev spaces that maps time into Banach Spaces (rather than just real number), and we may surprisingly get some "calculus"-like properties.

\begin{Def}
    The space $\mathscr{L}^p(0, T; X)$ consists of all strongly measurable functions $\textbf{u}: [0, T]\to X$ with 
    $$
        \|\textbf{u}\|_{\mathscr{L}^p(0, T;X)}:= \left(\int_0^T\|\textbf{u}(t)\|^pdt\right)^{1/p}<\infty
    $$ for $1\leq p<\infty$, and
    $$
    \|u\|_{\mathscr{L}^\infty(0,T; X)}:= \text{ess}\sup\limits_{0\leq t\leq T}\|\textbf{u}(t)\|<\infty
    $$
\end{Def}

\begin{Def}
    The space $\mathscr{C}([0, T]; X)$ comprises all continuous functions $\textbf{u}: [0, T]\to X$ with 
    $$
    \|\textbf{u}\|_{\mathscr{C}([0, T]; X)}:= \max_{0\leq t\leq T}\|\textbf{u}(t)\|<\infty
    $$
\end{Def}

\indent Next is the definition of weak derivative, which is almost identical to the usual one.

\begin{Def}
    Let $\textbf{u}\in\mathscr{L}^1(0,T; X)$. We say $\textbf{v}\in\mathscr{L}^1(0,T;X)$ is the weak derivative of $\textbf{u}$, written 
    $$
        \textbf{u}'=\textbf{v}
    $$
    provided 
    $$
    \int_0^T\phi'(t)\textbf{u}(t)dt=-\int_0^T\phi(t)\textbf{v}(t)dt
    $$ for all real valued test functions $\phi\in\mathscr{C}_c^\infty(0,T)$.
\end{Def}

\indent In light of the definition of weak derivative, we define Sobolev Spaces involving time.

\begin{Def}
    The Sobolev space $W^{1,p}(0, T; X)$ consists of all functions $\textbf{u}\in\mathscr{L}^p(0,T;X)$ such that $\textbf{u}'$ exists in the weak sense and belongs to $\mathscr{L}^p(0,T;X)$. Furthermore, 
    $$\|u\|_{\mathscr{C}([0,T];X)}:=\begin{cases}
  \left(\int_{0}^T\|\textbf{u}(t)\|^p+\|\textbf{u}'(t)\|^pdt\right)^{1/p}& (1\leq p<\infty) \\
  \text{ess}\sup\limits_{0\leq t\leq T}(\|\textbf{u}(t)\|+\|\textbf{u}'(t)\|)& (p=\infty)
    \end{cases}$$
\end{Def}

Done with all the definitions, we now observe that $W^{1,p}(0,T; X)$ can be naturally embedding in $\mathscr{C}(0,T; X)$ up to redefined on a measure zero set, and we can also do normal calculus operations (Fundamental Theorem of Calculus). 

\begin{Th}
    Let $\textbf{u}\in W^{1, p}(0,T; X)$ for some $1\leq p\leq\infty$. Then $\textbf{u}\in\mathscr{C}([0, T]; X)$ after possibly being redefined on a set of measure zero. Furthermore we have 
    $$
    \textbf{u}(t)=\textbf{u}(s)+\int_s^t\textbf{u}'(r)dr
    $$ for all $0\leq s\leq t\leq T$ and the estimate 
    $$
    \max_{0\leq t\leq T}\|\textbf{u}(t)\|\leq C\|\textbf{u}\|_{W^{1, p}(0, T; X)}
    $$ where the constant $C$ only depends on $T$.
\end{Th}

\begin{proof} The proof utilizes mollifier to turn $\textbf{u}$ into smooth function, where we are able to use Fundamental Theorem of Calculus. \\
In the proof, there are some technical details worth noting. First, Evans defines $\textbf{u}^\epsilon=\eta_{\epsilon}*\textbf{u}$ and claims that $\textbf{u}^\epsilon\to\textbf{u}$ in $\mathscr{L}^p(0,T;X)$. The reason is as follows, 
\begin{align*}
    \|\textbf{u}^\epsilon-\textbf{u}\|_{\mathscr{L}^p(0, T; X)}=\left(\int_0^T\|\textbf{u}^\epsilon(t)-\textbf{u}(t)\|^pdt\right)^{1/p}
\end{align*}
where 
\begin{align*}
    \|\textbf{u}^\epsilon(t)-\textbf{u}(t)\|&=\left\|\int_{(0, \epsilon)}\eta_{\epsilon}(y)\textbf{u}(t-y)dy-\int_{(0, \epsilon)}\eta_{\epsilon}(y)\textbf{u}(t)dy\right\| \\
    &=\left\|\int_{(0, \epsilon)}\eta_{\epsilon}(y)\left[\textbf{u}(t-y)-\textbf{u}(t)\right]dy\right\| \\
    &\leq \int_{(0, \epsilon)}\eta_{\epsilon}(y)\left\|\textbf{u}(t-y)-\textbf{u}(t)\right\|dy
\end{align*}
then 
\begin{align*}
    \int_0^T\left(\|\textbf{u}^\epsilon(t)-\textbf{u}(t)\|\right)^pdt&\leq\int_0^T\left(\int_{(0, \epsilon)}\eta_\epsilon(y)\|\textbf{u}(t-y)-\textbf{u}(t)\|dy\right)^pdt \\
    &\leq \int_{0}^TC\int_{(0, \epsilon)}\eta^p_\epsilon(y)\|\textbf{u}(t-y)-\textbf{u}(t)\|^pdydt \\
    &=C\int_{(0, \epsilon)}\eta^p_\epsilon(y)\int_0^T\|\textbf{u}(t-y)-\textbf{u}(t)\|^pdtdy 
\end{align*}
and by translation continuity of $\mathscr{L}^p$ function, we can control the inner integral by $\delta$ arbitrarily small by choosing $\epsilon$ small enough, then the upper bound will be an infinitesimal, thus obtaining the desired convergence. \\
\indent Finally, in the proof of the estimate, Evans says that it follows easily from 
$$
\textbf{u}(t)=\textbf{u}(s)+\int_{s}^t\textbf{u}'(\tau)d\tau
$$
The reason is as follows
\begin{align*}
    \|\textbf{u}(t)\|&\leq \|\textbf{u}(s)\|+\int_s^t\|\textbf{u}'(\tau)\|d\tau \leq \|\textbf{u}(s)\|+\int_0^t\|\textbf{u}'(\tau)d\|d\tau \\
    &\leq \|\textbf{u}(s)\|+T^{1-1/p}\|\textbf{u}'\|_{\mathscr{L}^p(0,T;X)} 
\end{align*}
by integrating over $s$, and notice that $(a+b)^p\leq 2^p(a^p+b^p)$, we may obtain 
\begin{align*}
    \int\|\textbf{u}(t)\|^pds\leq \int 2^p\left[\|\textbf{u}\|^p+T^{p-1}\|\textbf{u}'\|_{\mathscr{L}^p(0,T;X)}\right]ds
\end{align*}
then 
$$
T\|\textbf{u}(t)\|^p\leq 2^p\|\textbf{u}(s)\|^p_{\mathscr{L}^p(0,T;X)}+2^pT^p\|\textbf{u}'\|_{\mathscr{L}^p(0,T;X)}^p
$$ and the estimate follows easily.
\end{proof}

Evans also introduced other mappings into better spaces.

\begin{Th}
    Suppose $\textbf{u}\in \mathscr{L}^2(0,T;H_0^1(U))$, with $\textbf{u}'\in \mathscr{L}^2(0,T; H^{-1}(U))$. \\
    \indent 1. Then 
    $$
    \textbf{u}\in\mathscr{C}([0,T]; \mathscr{L}^2(U))
    $$ after being redefined on a set of measure zero. \\
    \indent 2. The mapping 
    $$
    t\mapsto \|\textbf{u}(t)\|_{\mathscr{L}^2(U)}^2
    $$ is absolutely continuous, with 
    $$
    \frac{d}{dt}\|\textbf{u}(t)\|^2_{\mathscr{L}^2(U)}=2\langle\textbf{u}'(t), \textbf{u}(t)\rangle
    $$ for a.e. $0\leq t\leq T$. \\
    \indent 3. Furthermore, we have the estimate 
    $$
    \max_{0\leq t\leq T} \|\textbf{u}(t)\|\leq C\left(\|\textbf{u}\|_{\mathscr{L}^2(0,T; H_0^1(U))}+\|\textbf{u}'\|_{\mathscr{L}^2(0,T; H^{-1}(U))}\right)
    $$ the constant $C$ depending on $T$.
\end{Th}

\begin{proof}
    The idea of the proof is to first extend the function $\textbf{u}$ to a slightly larger domain of $[-\sigma,T+\sigma]$, where we can use mollifier to regularize $\textbf{u}$, then use Fundamental Theorem of Calculus on Banach Spaces. \\
    \indent There are two parts worth mentioning. First, in the proof of (i), Evans says that 
    $$
    \|\textbf{u}^\epsilon(t)-\textbf{u}^\delta(t)\|_{\mathscr{L}^2(U)}^2=\|\textbf{u}^\epsilon(s)-\textbf{u}^\delta(s)\|_{\mathscr{L}^2(U)}^2+2\int_s^t\langle\textbf{u}^{\epsilon'}(\tau)-\textbf{u}^{\delta'}(\tau), \textbf{u}^\epsilon(\tau)-\textbf{u}^\delta(\tau)\rangle d\tau
    $$ implies 
    \begin{align*}
        \limsup_{\epsilon,\delta\to0} \sup_{0\leq t\leq T} \|\textbf{u}^\epsilon(t)-\textbf{u}^\delta(t)\|_{\mathscr{L}^2(U)}^2 \leq \lim_{\epsilon,\delta\to 0}\int_0^T\|\textbf{u}^{\epsilon'}(\tau)-\textbf{u}^{\delta'}(\tau)\|_{H^{-1}(U)}^2 \\
        +\|\textbf{u}^\epsilon(\tau)-\textbf{u}^\delta(\tau)\|_{H_0^1(U)}^2d\tau
    \end{align*}
    The reason is as follows: Since 
    \begin{align*}
        \left|\langle\textbf{u}^{\epsilon'}(\tau)-\textbf{u}^{\delta'}(\tau), \textbf{u}^\epsilon(\tau)-\textbf{u}^\delta(\tau)\rangle \right|\leq \|\textbf{u}^{\epsilon'}(\tau)-\textbf{u}^{\delta'}(\tau)\|_{H^{-1}(U)}\cdot \|\textbf{u}^\epsilon(\tau)-\textbf{u}^\delta(\tau)\|_{H_0^1(U)}
    \end{align*}
    thus 
    \begin{align*}
        \int_s^t\langle\textbf{u}^{\epsilon'}(\tau)-\textbf{u}^{\delta'}(\tau), \textbf{u}^\epsilon(\tau)-\textbf{u}^\delta(\tau)\rangle d\tau\leq \int_s^t \|\textbf{u}^{\epsilon'}(\tau)-\textbf{u}^{\delta'}(\tau)\|_{H^{-1}(U)} \\
        \cdot \|\textbf{u}^\epsilon(\tau)-\textbf{u}^\delta(\tau)\|_{H_0^1(U)}d\tau
    \end{align*}
    then 
    \begin{align*}
        \|\textbf{u}^\epsilon(t)-\textbf{u}^\delta(t)\|_{\mathscr{L}^2(U)}^2\leq \|\textbf{u}^\epsilon(s)-\textbf{u}^\delta(s)\|_{\mathscr{L}^2(U)}^2 + \int_s^t \|\textbf{u}^{\epsilon'}(\tau)-\textbf{u}^{\delta'}(\tau)\|_{H^{-1}(U)} \\
        \cdot \|\textbf{u}^\epsilon(\tau)-\textbf{u}^\delta(\tau)\|_{H_0^1(U)}d\tau 
    \end{align*}
    Since we know 
    \begin{align*}
        \|\textbf{u}^{\epsilon'}(\tau)-\textbf{u}^{\delta'}(\tau)\|_{H^{-1}(U)}
        \cdot \|\textbf{u}^\epsilon(\tau)-\textbf{u}^\delta(\tau)\|_{H_0^1(U)}\leq \frac{1}{2}\|\textbf{u}^{\epsilon'}(\tau)-\textbf{u}^{\delta'}(\tau)\|_{H^{-1}(U)}^2+ \\
        \frac{1}{2}\|\textbf{u}^\epsilon(\tau)-\textbf{u}^\delta(\tau)\|_{H_0^1(U)}^2
    \end{align*}
    By the fact that $\textbf{u}^\epsilon(s)\to\textbf{u}(s)$ in $\mathscr{L}^2$ for a.e. $s\in(0,T)$, we may obtain the desired conclusion. \\
    \indent Finally, in order to obtain the estimate in (iii), Evans says we should integrate 
    $$
    \|\textbf{u}(t)\|_{\mathscr{L}^2(U)}^2=\|\textbf{u}(s)\|_{\mathscr{L}^2}^2+2\int_s^t\langle\textbf{u}'(\tau), \textbf{u}(\tau)\rangle d\tau 
    $$ and do some simple estimates, to be more specific, we calculate
    \begin{align*}
        T\|\textbf{u}(t)\|_{\mathscr{L}^2(U)}^2=\int_0^T\|\textbf{u}(s)\|_{\mathscr{L}^2(U)}^2ds+2\int_0^T\int_s^t\langle\textbf{u}'(\tau), \textbf{u}(\tau)\rangle d\tau ds
    \end{align*}
    where 
    \begin{align*}
        \int_0^T\int_s^t \langle\textbf{u}'(\tau), \textbf{u}(\tau)\rangle d\tau ds&=\int_s^t\int_0^T \langle\textbf{u}'(\tau), \textbf{u}(\tau)\rangle dsd\tau \\
        &= \int_0^t\int_0^\tau\langle\textbf{u}'(\tau), \textbf{u}(\tau)\rangle dsd\tau + \int_t^T\int_\tau^T \langle\textbf{u}'(\tau), \textbf{u}(\tau)\rangle dsd\tau \\
        &=\int_0^t \tau \langle\textbf{u}'(\tau), \textbf{u}(\tau)\rangle d\tau +\int_t^T (T-\tau) \langle\textbf{u}'(\tau), \textbf{u}(\tau)\rangle d\tau \\
        &\leq T\int_0^t \langle\textbf{u}'(\tau), \textbf{u}(\tau)\rangle d\tau + T\int_t^T \langle\textbf{u}'(\tau), \textbf{u}(\tau)\rangle d\tau \\
        &= T\int_0^T \langle\textbf{u}'(\tau), \textbf{u}(\tau)\rangle d\tau
    \end{align*}
    Recall the estimate $\langle\textbf{u}'(\tau), \textbf{u}(\tau)\rangle\leq \|\textbf{u}'(\tau)\|_{H^{-1}(U)}\cdot\|\textbf{u}(\tau)\|_{H_0^1(U)}$, then
    \begin{align*}
        2\int_0^T\int_s^t \langle\textbf{u}'(\tau), \textbf{u}(\tau)\rangle d\tau ds&\leq 2T\int_0^T \|\textbf{u}'(\tau)\|_{H^{-1}(U)}\cdot\|\textbf{u}(\tau)\|_{H_0^1(U)} d\tau \\
        &\leq 2T\left(\int_0^T \|\textbf{u}'(\tau)\|_{H^{-1}(U)}^2d\tau\right)^{1/2}\cdot\left(\int_0^T\|\textbf{u}(\tau)\|_{H_0^1(U)}^2d\tau \right)^{1/2} \\
        &=:2T XY
    \end{align*}
    Notice that $\|\textbf{u}(s)\|_{\mathscr{L}^2(U)}\leq\|\textbf{u}(s)\|_{H_0^1}$, and the original equation becomes:
    \begin{align*}
        T\|\textbf{u}(t)\|_{\mathscr{L}^2(U)}^2\leq Y^2 + 2TXY\leq (Y+TX)^2
    \end{align*}
    then
    \begin{align*}
        \|\textbf{u}(t)\|^2_{\mathscr{L}^2(U)}\leq \frac{Y}{\sqrt{T}}+\sqrt{T}X\leq C(T)(X+Y)
    \end{align*}
    where we simply let $C(T)=\max\{\sqrt{T}, 1/\sqrt{T}\}$
\end{proof}

\begin{Th}
    Assume that $U$ is open, bounded, and $\partial U$ is smooth. Take $m$ to be a nonnegative integer. \\
    \indent Suppose $\textbf{u}\in\mathscr{L}^2(0,T; H^{m+2}(U))$, with $\textbf{u}'\in\mathscr{L}^2(0,T;H^{m+1}(U))$. \\
    \indent (i) Then 
    $$
    \textbf{u}\in\mathscr{C}([0,T]; H^{m+1}(U)) 
    $$ after redefining on a measure zero set. \\
    \indent (ii) Furthermore, we have the estimate 
    $$
    \max_{0\leq t\leq T}\|\textbf{u}(t)\|_{H^{m+1}(U)}\leq C\left(\|\textbf{u}\|_{\mathscr{L}^2(0,T; H^{m+2}(U))}+\|\textbf{u}'\|_{\mathscr{L}^2(0,T; H^m(U))}\right)
    $$ where the constant depends only on $T,U,m$.
\end{Th}

\begin{proof}
    The general idea of the proof is to utilize difference quotients introduced in 5.8.2 and Theorem 3 (Difference quotients ensures weak derivative and its norm can be controlled) in that section. \\
    \indent However, when applying it, notice that that theorem apply only on a compactly contained subset of the domain, then we naturally think of using Extension Theorem to first extend our function $\textbf{u}$ by a bit, and the rest of the proof should be clear from the text.
\end{proof}

\end{document}