\documentclass{article}
\usepackage[utf8]{inputenc}
\usepackage{amsthm}
\usepackage{amsmath}
\usepackage{amssymb}
\usepackage{mathrsfs}
\usepackage{stix}

\title{Notes on Evans PDE, Chapter 5}
\author{Yunzhe Zheng}
\date{2025}

\begin{document}

\maketitle

\section{On Sobolev embeddings}

\indent\indent We observe various inequalities involving $u\in W^{k, p}$ and other more regular spaces ($\mathscr{L}^q$, Hölder Spaces), where the Sobolev norm acts as the upper-bound, to show how Sobolev spaces can be included (embedded) in these spaces.

\subsection{Gagliardo-Nirenberg-Sobolev inequality}

\indent\indent This inequality aims at embedding $W^{1, p}$ into certain $\mathscr{L}^{q}$. Notice that $p$ should be in $[1, n)$. \\
\indent The idea of the proof is to prove the ineqaulity for $\mathscr{C}^1_c$, then utilize the extension Theorem and approximation Theorem introduced before to obtain the desired result. 

\subsubsection{Case of $u\in\mathscr{C}^1_{c}$}

\newtheorem{Th}{Theorem}

\begin{Th}
    Assume $1\leq p<n$. There exists a constant $C$, depending only on $p$ and $n$, such that
    $$
    \|u\|_{\mathscr{L}^{p^*}(\mathbb{R}^{n})}\leq C\|Du\|_{L^{p}(\mathbb{R}^{n})}
    $$
    for all $u\in\mathscr{C}^{1}_{c}(\mathbb{R}^n)$.
\end{Th}

\begin{proof}
    Sketch: First proof for $p=1$ by inductively using Hölder inequality, then consider $v=|u|^{\gamma}$, where $\gamma=\frac{p(n-1)}{n-p}>1$, and estimate 
    $$
    \left(\int_{\mathbb{R}^n}|u|^{\frac{\gamma n}{n-1}}dx\right)^{\frac{n}{n-1}}
    $$
\end{proof}
The proof uses Hölder's inequality secretly and intensively, to give one example, let's take a look at the inequality under $(12)$: 

\begin{align*}
    &\left(\int_{-\infty}^{\infty}|Du|dy_1\right)^{\frac{1}{n-1}}\int_{-\infty}^{\infty}\prod_{i=2}^{n}\left(\int_{-\infty}^{\infty}|Du|dy_i\right)^{\frac{1}{n-1}}dx_1 \\
    \leq& \left(\int_{-\infty}^{\infty}|Du|dy_1\right)^{\frac{1}{n-1}}\left(\prod_{i=2}^{n}\int_{-\infty}^{\infty}\int_{-\infty}^{\infty}|Du|dx_1dy_i\right)^{\frac{1}{n-1}}
\end{align*}
where we take $f_i=\left(\int_{-\infty}^{\infty}|Du|dy_i\right)^{\frac{1}{n-1}}\in L^{n-1}$ for $i=2, \dots, n$, and apply Hölder inequality directly. Eventually we may get
$$
\int_{\mathbb{R}^n}|u|^{\frac{n}{n-1}}dx\leq\left(\int_{\mathbb{R}^n}|Du|dx\right)^{\frac{n}{n-1}}
$$

\subsubsection{Case of $u\in W^{1,p}$, $1\leq p< n$}

\begin{Th}
    Let $U$ be a bounded, open subset of $\mathbb{R}^n$, and suppose that $\partial U$ is $\mathscr{C}^1$. Assume $1\leq p<n$, and $u\in W^{1, p}(U)$. Then $u\in \mathscr{L}^{p^*}$, with the estimate $\|u\|_{\mathscr{L}^{p^*}}\leq C\|u\|_{W^{1, p}(U)}$.
\end{Th}

\begin{proof}
    Sketch: Apply the Extension Theorem to extend $u$ from $U$ to $\mathbb{R}^n$, where the extended function $\overline{u}=u$ in $U$ and has compact support, and also by Approximation Theorem, we can find a sequence of $\mathscr{C}^{\infty}_{c}(\mathbb{R}^n)$ functions such that $u_m\to\overline{u}$ in $W^{1, p}(\mathbb{R}^n)$.\\
    \indent Previous Theorem implies following two estimates $\|u_m-u_l\|_{\mathscr{L}^{p^*}}\leq C\|Du_m-Du_l\|_{\mathscr{L}^{p}(\mathbb{R}^n)}$ and $\|u_{m}\|_{\mathscr{L}^{p^*}(\mathbb{R}^n)}\leq C\|Du_m\|_{\mathscr{L}^{p}(\mathbb{R}^n)}$ (1), where the first estimate yields $u_m\to\overline{u}$ in $\mathscr{L}^{p^*}$, thus we have $\|u_m\|_{\mathscr{L}^{p^*}}\to\|\overline{u}\|_{\mathscr{L}^{p^*}} (2)$ by the fact that $|\|u_m\|_{\mathscr{L}^{p^*}}-\|\overline{u}\|_{\mathscr{L}^{p^*}}|\leq \|u_m-\overline{u}\|_{\mathscr{L}^{p^*}}$. Also, convergence in $W^{1,p}$ implies that $\|Du_m-D\overline{u}\|_{\mathscr{L}^p}$, thus $\|Du_m\|_{\mathscr{L}^{p}}\to\|D\overline{u}\|_{\mathscr{L}^{p}}$ (3). Combining (1) (2) (3) we obtain the desired result.
\end{proof}

One remark worth noting about the application of Extension Theorem is that although in the statement of Extension Theorem it is not explicitly mentioned that the extended function is compactly supported, but actually in our context it is automatic. 

\newtheorem{Def}{Definition}[section]
\begin{Def}
    The Support a function is the smallest closed set containing all points that are not mapped to $0$.
\end{Def}

Since in our context, the domain is always bounded, thus the support is always compact.

\subsubsection{Case of $u\in W^{1,p}_{0}$, $1\leq p< n$}

\indent\indent The advantage of $W^{1,p}_0$ is that not only can we embed it into $\mathscr{L}^{p^*}$ for specific $p^*$ depending on $p$, but also we can embed it into all $\mathscr{L}^{q}$ for $1\leq q\leq p^*$.

Refer to textbook for the exact statement and proof.

\subsection{Morrey's inequality}

\indent\indent Unlike the G-N-S inequality, Morrey's inequality aims at solving the embedding problem for $n<p\leq \infty$ into Holder space.

\subsubsection{Case of $u\in \mathscr{C}^{1}$}

\newtheorem{Lem}{Lemma}
\begin{Lem}
    There exists a constant $C$ depending on $n$ such that 
    $$
    \intbar_{B(x,r)}|u(y)-u(x)|dy\leq\int_{B(x,r)}\frac{|Du(y)|}{|y-x|^{n-1}}dy
    $$
    for $u\in\mathscr{C}^1(\mathbb{R}^n)$.
\end{Lem} 

This is very important estimate, see text for proof. \\

I shall not repeat the statement of Morrey's inequality here, but there is another important estimate in the proof that worth mentioning:
$$
    |u(x)-u(y)|\leq Cr^{1-\frac{n}{p}}\|u\|_{\mathscr{L}^p(\mathbb{R}^n)}
$$ where $r=|x-y|$. In the remark after Morrey's inequality, the author claims another similar estimate without proof
$$
    |u(y)-u(x)|\leq Cr^{1-\frac{n}{p}}\left(\int_{B(x,2r)}|Du(z)|^p\right)^{1/p}
$$

The proof of this estimate is actually immediate once we replace all $r$ by $2r$ in the proof of the above estimate, see text for reference.

\subsubsection{Case of $u\in W^{1, p}$, $n< p\leq \infty$}

\indent\indent We again utilize Extension Theorem and Approximtion Theorem, the proof is almost identical to the case of G-N-S inequality.

\subsection{Borderline Case of $p=n$}

\indent\indent This part resolves the problem of the boundary case where $p=n$, which I believe Evans book didn't discuss explicitly. The book I'm going to refer to is "Functional Analysis, Sobolev Spaces and Partial Differential Equations" by Haim Brezis.

To achieve the border case, we need a more sophisticated estimate:

\begin{Lem}
    Let $N\geq 2$ and let $f_1, f_2, \dots, f_N\in\mathscr{L}^{N-1}(\mathbb{R}^{N-1})$. For $x\in\mathbb{R}^N$ and $1\leq i\leq N$ set
    $$
    \tilde{x}_{i} = (x_1, x_2, \dots, x_{i-1}, x_{i+1}, \dots, x_N)\in\mathbb{R}^{N-1}
    $$
    then the function $f(x)=f_{1}(\tilde{x}_1)f_2(\tilde{x}_2)\dots f_N(\tilde{x}_N)$, $x\in\mathbb{R}^N$ belongs to $\mathscr{L}^1(\mathbb{R}^n)$ and with the inequality 
    $$    \|f\|_{\mathscr{L}^1(\mathbb{R}^N)}\leq\prod_{i=1}^N\|f_i\|_{\mathscr{L}^{N-1}(\mathbb{R}^{N-1})}
    $$.
\end{Lem}

\begin{proof}
    The case of $N=2$ is simply Minkowski inequality. For $N=3$, we have 
    \begin{align*}
        \int_{\mathbb{R}}|f(x)|dx_3 &= |f_{3}(x_1, x_2)|\int_{\mathbb{R}}|f_1(x_2, x_3)||f_{2}(x_1, x_3)|dx_3  \\
        &\leq |f_3(x_1, x_2)|\left(\int_{\mathbb{R}}|f_{1}(x_2, x_3)|^2dx_3\right)^{1/2}\left(\int_{\mathbb{R}}|f_{2}(x_1, x_3)|^2dx_3\right)^{1/2} \\
    \end{align*}
    Integrating over $x_1, x_2$ and apply Cauchy-Schwarz inequality
    \begin{align*}
    &\int_{\mathbb{R}}\int_{\mathbb{R}}|f(x)|dx_3dx_1\leq \left(\int_{\mathbb{R}}|f_{1}(x_2, x_3)|^2dx_3\right)^{1/2}\int_\mathbb{R}|f_3(x_1, x_2)|\left(\int_{\mathbb{R}}|f_{2}(x_1, x_3)|^2dx_3\right)^{1/2}dx_1 \\
    &\leq \left(\int_{\mathbb{R}}|f_{1}(x_2, x_3)|^2dx_3\right)^{1/2}\left(\int_{\mathbb{R}}|f_3(x_1, x_2)|^2dx_1\right)^{1/2}\left(\int_{\mathbb{\mathbb{R}}}\int_{\mathbb{R}}|f_2(x_1, x_3)|^2dx_3dx_1\right)^{1/2}
    \end{align*}
    Then 
    \begin{align*}
    \int_{\mathbb{R}}\int_{\mathbb{R}}\int_{\mathbb{R}}|f(x)|dx_3dx_1dx_2&\leq\left(\int_{\mathbb{R}^2}|f_2(x_1, x_3)|^2dx_1dx_3\right)^{1/2}\left(\int_{\mathbb{R}}\int_{\mathbb{R}}|f_1(x_2, x_3)|^2dx_2dx_3\right)^{1/2} \\
    &\left(\int_{\mathbb{R}}\int_{\mathbb{R}}|f_3(x_1, x_2)|dx_1dx_2\right)^{1/2}
    \end{align*} which finish the case of $N=3$. \\
    For general $N$, we use induction and prove for $N+1$
    \begin{align}
    \int_{\mathbb{R}^{N+1}}|f(x)|dx\leq \|f_{N+1}\|_{\mathscr{L}^N(\mathbb{R}^N)}\left[\int_{\mathbb{R}^N}|f_{1}f_{2}\cdots f_{N}|^{N'}dx_1dx_2\dots dx_N\right]^{1/N'}
    \end{align}
    where $N'=N/(N-1)$. Then by induction step, we have
    \begin{align}
    \int_{\mathbb{R}^N}|f_1|^{N'}\cdots|f_N|^{N'}dx_1\dots dx_N\leq \prod_{i=1}^{N}\left\||f_{i}|^{N'}\right\|_{\mathscr{L}^{N-1}(\mathbb{R}^{N-1})}=\prod_{i=1}^{N}\|f_{i}\|^{N'}_{\mathscr{L}^{N}(\mathbb{R}^{N-1})}
    \end{align}
    where the last inequality follows from 
    $$
    \left[\int_{\mathbb{R}^{N-1}}\left(|f_i|^{N/N-1}\right)^{N-1}\right]^{\frac{1}{N-1}}=\left(\int_{\mathbb{R}^{N-1}}|f_i|^{N}\right)^{\frac{1}{N}\cdot\frac{N}{N-1}}=\|f_i\|_{\mathscr{L}^{N}(\mathbb{R}^{N-1})}^{N'}
    $$
    \indent Notice that in the induction step, the author didn't prove that $|f_i|^{N'}$ is in $\mathscr{L}^{N-1}$, which is a crucial criteria for induction, but that's immediate, as we see
    $$
    \left[\int_{\mathbb{R}^{N-1}}\left(|f_i|^{\frac{N}{N-1}}\right)^{N-1}\right]^{\frac{1}{N-1}}=\left[\int_{\mathbb{R}^{N-1}}|f_i|^N\right]^{\frac{1}{N-1}}<\infty
    $$
    Plug estimate $(2)$ into estimate $(1)$ and we are done.
\end{proof} 

Now we are heading to the more sophisticated estimate that eventually leads us to the borderline case.

\begin{Lem}
    Let $m\geq 1$, and $u\in\mathscr{C}^{1}_{c}(\mathbb{R}^n)$, we have following estimate
    $$
    \|u\|_{mN/(N-1)}^{m}\leq m\prod_{i=1}^{N}\left\||u|^{m-1}\frac{\partial u}{\partial x_i}\right\|_{1}^{1/N}
    $$
\end{Lem}

\subsection{General Sobolev inequality}

\indent\indent In this section, we generalize our embedding by considering not only $W^{1, p}$ spaces but $W^{k,p}$ spaces, an in turn we get (as expected) more complicated embeddings. 



\end{document}