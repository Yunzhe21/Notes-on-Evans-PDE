\documentclass{article}
\usepackage[utf8]{inputenc}
\usepackage{amsthm}
\usepackage{amsmath}
\usepackage{amssymb}
\usepackage{mathrsfs}
\usepackage{stix}

\title{Notes on Evans PDE, Chapter 5}
\author{Yunzhe Zheng}
\date{2025}

\begin{document}

\maketitle

\section{On Sobolev embeddings}

\indent\indent We observe various inequalities involving $u\in W^{k, p}$ and other more regular spaces ($\mathscr{L}^q$, Hölder Spaces), where the Sobolev norm acts as the upper-bound, to show how Sobolev spaces can be included (embedded) in these spaces.

\subsection{Gagliardo-Nirenberg-Sobolev inequality}

\indent\indent This inequality aims at embedding $W^{1, p}$ into certain $\mathscr{L}^{q}$. Notice that $p$ should be in $[1, n)$. \\
\indent The idea of the proof is to prove the ineqaulity for $\mathscr{C}^1_c$, then utilize the extension Theorem and approximation Theorem introduced before to obtain the desired result. 

\subsubsection{Case of $u\in\mathscr{C}^1_{c}$}

\newtheorem{Th}{Theorem}

\begin{Th}
    Assume $1\leq p<n$. There exists a constant $C$, depending only on $p$ and $n$, such that
    $$
    \|u\|_{\mathscr{L}^{p^*}(\mathbb{R}^{n})}\leq C\|Du\|_{L^{p}(\mathbb{R}^{n})}
    $$
    for all $u\in\mathscr{C}^{1}_{c}(\mathbb{R}^n)$.
\end{Th}

\begin{proof}
    Sketch: First proof for $p=1$ by inductively using Hölder inequality, then consider $v=|u|^{\gamma}$, where $\gamma=\frac{p(n-1)}{n-p}>1$, and estimate 
    $$
    \left(\int_{\mathbb{R}^n}|u|^{\frac{\gamma n}{n-1}}dx\right)^{\frac{n}{n-1}}
    $$
\end{proof}
The proof uses Hölder's inequality secretly and intensively, to give one example, let's take a look at the inequality under $(12)$: 

\begin{align*}
    &\left(\int_{-\infty}^{\infty}|Du|dy_1\right)^{\frac{1}{n-1}}\int_{-\infty}^{\infty}\prod_{i=2}^{n}\left(\int_{-\infty}^{\infty}|Du|dy_i\right)^{\frac{1}{n-1}}dx_1 \\
    \leq& \left(\int_{-\infty}^{\infty}|Du|dy_1\right)^{\frac{1}{n-1}}\left(\prod_{i=2}^{n}\int_{-\infty}^{\infty}\int_{-\infty}^{\infty}|Du|dx_1dy_i\right)^{\frac{1}{n-1}}
\end{align*}
where we take $f_i=\left(\int_{-\infty}^{\infty}|Du|dy_i\right)^{\frac{1}{n-1}}\in L^{n-1}$ for $i=2, \dots, n$, and apply Hölder inequality directly. Eventually we may get
$$
\int_{\mathbb{R}^n}|u|^{\frac{n}{n-1}}dx\leq\left(\int_{\mathbb{R}^n}|Du|dx\right)^{\frac{n}{n-1}}
$$

\subsubsection{Case of $u\in W^{1,p}$, $1\leq p< n$}

\begin{Th}
    Let $U$ be a bounded, open subset of $\mathbb{R}^n$, and suppose that $\partial U$ is $\mathscr{C}^1$. Assume $1\leq p<n$, and $u\in W^{1, p}(U)$. Then $u\in \mathscr{L}^{p^*}$, with the estimate $\|u\|_{\mathscr{L}^{p^*}}\leq C\|u\|_{W^{1, p}(U)}$.
\end{Th}

\begin{proof}
    Sketch: Apply the Extension Theorem to extend $u$ from $U$ to $\mathbb{R}^n$, where the extended function $\overline{u}=u$ in $U$ and has compact support, and also by Approximation Theorem, we can find a sequence of $\mathscr{C}^{\infty}_{c}(\mathbb{R}^n)$ functions such that $u_m\to\overline{u}$ in $W^{1, p}(\mathbb{R}^n)$.\\
    \indent Previous Theorem implies following two estimates $\|u_m-u_l\|_{\mathscr{L}^{p^*}}\leq C\|Du_m-Du_l\|_{\mathscr{L}^{p}(\mathbb{R}^n)}$ and $\|u_{m}\|_{\mathscr{L}^{p^*}(\mathbb{R}^n)}\leq C\|Du_m\|_{\mathscr{L}^{p}(\mathbb{R}^n)}$ (1), where the first estimate yields $u_m\to\overline{u}$ in $\mathscr{L}^{p^*}$, thus we have $\|u_m\|_{\mathscr{L}^{p^*}}\to\|\overline{u}\|_{\mathscr{L}^{p^*}} (2)$ by the fact that $|\|u_m\|_{\mathscr{L}^{p^*}}-\|\overline{u}\|_{\mathscr{L}^{p^*}}|\leq \|u_m-\overline{u}\|_{\mathscr{L}^{p^*}}$. Also, convergence in $W^{1,p}$ implies that $\|Du_m-D\overline{u}\|_{\mathscr{L}^p}$, thus $\|Du_m\|_{\mathscr{L}^{p}}\to\|D\overline{u}\|_{\mathscr{L}^{p}}$ (3). Combining (1) (2) (3) we obtain the desired result.
\end{proof}

One remark worth noting about the application of Extension Theorem is that although in the statement of Extension Theorem it is not explicitly mentioned that the extended function is compactly supported, but actually in our context it is automatic. 

\newtheorem{Def}{Definition}[section]
\begin{Def}
    The Support a function is the smallest closed set containing all points that are not mapped to $0$.
\end{Def}

Since in our context, the domain is always bounded, thus the support is always compact.

\subsubsection{Case of $u\in W^{1,p}_{0}$, $1\leq p< n$}

The advantage of $W^{1,p}_0$ is that not only can we embed it into $\mathscr{L}^{p^*}$ for specific $p^*$ depending on $p$, but also we can embed it into all $\mathscr{L}^{q}$ for $1\leq q\leq p^*$.

Refer to textbook for the exact statement and proof.

\subsection{Morrey's inequality}

Unlike the G-N-S inequality, Morrey's inequality aims at solving the embedding problem for $n<p\leq \infty$ into Holder space.

\subsubsection{Case of $u\in \mathscr{C}^{1}$}

\newtheorem{Lem}{Lemma}
\begin{Lem}
    There exists a constant $C$ depending on $n$ such that 
    $$
    \intbar_{B(x,r)}|u(y)-u(x)|dy\leq\int_{B(x,r)}\frac{|Du(y)|}{|y-x|^{n-1}}dy
    $$
    for $u\in\mathscr{C}^1(\mathbb{R}^n)$.
\end{Lem} 

This is very important estimate, see text for proof.

I shall not repeat the statement of Morrey's inequality here, but there is one 
\subsubsection{Case of $u\in W^{1, p}$, $n< p\leq \infty$}

\subsection{Boundary Case of $p=n$}
\end{document}